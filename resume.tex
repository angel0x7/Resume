\documentclass[a4paper,10pt]{article}
\usepackage[utf8]{inputenc}
\usepackage[T1]{fontenc}
\usepackage[french, english]{babel}
\usepackage[hidelinks]{hyperref}
\usepackage{geometry}
\usepackage{enumitem}
\usepackage{xcolor}
\usepackage{titlesec}
\usepackage{parskip} % espace entre paragraphes
\geometry{left=0.5cm,right=0.5cm,top=1cm,bottom=0.5cm} % marges réduites

% Couleurs
\definecolor{myblue}{HTML}{003366}

% Format section
\titleformat{\section}{
  \normalfont\large\bfseries\color{myblue}
}{}{0em}{}[\titlerule]

% Commande pour aligner date à droite dans une ligne
\newcommand{\dates}[1]{\hfill {\footnotesize \textit{#1}}}

% Commande pour titre avec lien
\newcommand{\linkedtitle}[2]{\href{#1}{\textbf{\color{myblue}#2}}}

% Réduire espace entre items
\setlist[itemize]{noitemsep, topsep=2pt}

\begin{document}

% ======= NOM =======
\begin{center}
    {\LARGE \textbf{Angel Velasco}}\\[2pt]
    {\small \textbf{Étudiant ingénieur M1 Cybersécurité}}\\[5pt]
     
    \small
    \href{mailto:angel.velasco0x7@gmail.com}{angel.velasco0x7@gmail.com} \quad
    +33 6 37 73 62 81 \quad
    Yvelines, France \\
    {\small \href{https://angel0x7.github.io}{Portfolio } \href{https://www.linkedin.com/in/angel-velasco-0x7/}{Linkedin}} 
\end{center}
    



% ======= PROFIL =======


\begin{center}
Je recherche un \textbf{stage technique de 4 mois dès avril 2026} en cybersécurité défensive.  
Intéressé par les \textbf{réseaux}, le \textbf{forensic}, le \textbf{DevOps SecOps} et les \textbf{enjeux de chiffrement} en entreprise.

\end{center}




% ======= EDUCATION =======
\section*{Formation}

\noindent
\linkedtitle{https://www.ece.fr/en/program/engineering-cycle-bac4-information-systems-and-cybersecurity-major/}{École Centrale d'Électronique} \hfill Paris, France \\
\emph{Diplôme d'ingénieur – Majeure Cybersécurité – Mineures Défense-Sécurité / International} \hfill Septembre 2025 -- Présent \\
Spécialisation en cybersécurité défensive développée en partenariat avec Microsoft et Orange Cyberdefense,\\ certifiée SecNumEdu par l’ANSSI. 

\noindent
\textbf{\href{https://www.linkedin.com/company/asso0xece/}{Membre — Cyber 0xECE}} : ateliers, CTF, projets en cybersécurité. \\
\textbf{Security Blue Team} — Blue Team Junior Analyst \hfill Juin 2025 -- Juillet 2025 \\
\textbf{Fortinet} — Certified Fundamentals \& Associate (FCF + FCA)\hfill Juin 2025 -- Juillet 2025 


% ======= COMPETENCES & PROJETS =======
\section*{Projets}


\begin{itemize}[leftmargin=*]


\item \href{https://github.com/angel0x7/IoT-Adversary-Emulator}{\textbf{Prototype Outil d'Émulation d'Adversaire IoT - En cours}} \\
Conception et implémentation d'une plateforme de test pour simuler scénarios d'attaque IoT (reconnaissance, injection MQTT, replay, DoS, compromission de capteurs) afin d'évaluer la résilience des dispositifs et d'entraîner des détecteurs d'anomalies.



  \item \href{https://github.com/angel0x7/Honeypot-}{\textbf{Honeypot}} \\
  Cybersécurité défensive : déploiement d’un honeypot sur Azure. Analyse des tentatives d’attaque et production d’événements dans Microsoft Sentinel, analyse KQL, enrichissement géolocalisé.

  \item \href{https://github.com/angel0x7/Firewall-Lab-WAN-LAN-DMZ-}{\textbf{Firewall-Lab (WAN/LAN/DMZ)}} \\
  Conception et déploiement d’un laboratoire virtuel reproduisant une architecture réseau d’entreprise, intégrant trois zones distinctes (WAN, LAN, DMZ) via pfSense.
  \item \href{https://github.com/angel0x7/Detection-AI-models-}{\textbf{Detection-AI-models}} \\
  Détection du trafic réseau suspect (intrusions, attaques DoS, scans)
  \item \href{https://github.com/angel0x7/Web_site_java}{\textbf{Site Web}} \\
Développement d’une application Java simulant un panier d’achat avec gestion des remises, facturation et archivage des ventes. Projet réalisé en ING3 à l’ECE Paris, intégrant logique métier et persistance des données.



\end{itemize}

\noindent
\textbf{GitHub :} \href{https://github.com/angel0x7}{github.com/angel0x7}


% ======= PROFESSIONAL EXPERIENCE =======
\section*{Expérience Professionnelle}
\begin{itemize}[leftmargin=*]

  \item
  \textbf{\href{https://www.renaultgroup.com/}{Technicien de Maintenance}}\hfill \emph{Douai, France} \
  
  \emph{Renault Group} \hfill Janvier 2025 – Février 2025 \
  \begin{itemize}
     \item Analyse de données issues de capteurs industriels pour anticiper les défaillances et garantir la disponibilité.  
    \item Contribution à la sensibilisation en \textbf{cybersécurité industrielle} : SCADA, réseaux OT, prévention d’intrusions et plan de continuité.  
    \item Collaboration avec l’équipe maintenance pour renforcer la sécurité et la fiabilité des systèmes.    
  \end{itemize}

  \item
  \textbf{\href{https://www.inmapa.com/en/}{Opérateur (Stage)}} \hfill \emph{Palencia, Espagne} \
  
  \emph{INMAPA} \hfill Janvier 2024 – Février 2024 \
  \begin{itemize}
    \item Réalisation de câblages et mise en service de systèmes industriels dans un environnement exigeant en conformité et sécurité.  
    \item Participation à la \textbf{sécurisation d’armoires électriques et de systèmes industriels }.  
    \item Respect strict des normes de sécurité aéronautiques et tests de conformité.  
 
  \end{itemize}

  \item
  \textbf{Stagiaire Ingénierie Process} \hfill \emph{Guyancourt, France} \

  \emph{Renault Group} \hfill Juillet 2021 – Août 2021 \
  \begin{itemize}
     
    \item Participation à l’optimisation des procédés et sensibilisation aux contraintes de fiabilité et de sécurité.  
    
  \end{itemize}

\end{itemize}




% ======= COMPETENCES =======
\section*{Compétences}

\begin{tabular}{p{4.5cm} p{15cm}}
\textbf{Cybersécurité \& Réseaux} & pfSense (WAN/LAN/DMZ), honeypot, Azure Sentinel, SCADA/OT, MITRE ATT\&CK, NIST, Proxy / Reverse Proxy. \\
\textbf{Protocoles} & TCP/IP, UDP, SSH, TLS/SSL, HTTP/HTTPS, DNS. \\
\textbf{Systèmes} & Windows Server,Active Directory , Linux (sécurité et exploitation). \\
\textbf{Programmation} & Python, C/C++, Java , JavaScript (API), SQL. \\

\textbf{Cloud \& DevOps} & Microsoft Azure, CI/CD, pipelines DevOps, VirtualBox, Docker. \\

\textbf{Gestion de projet} & GitHub, Trello, Gantt, PERT. \\
\textbf{Langues} & Français (natif), Espagnol (natif), Anglais (B2). \\
\end{tabular}




\end{document}
