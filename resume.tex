\documentclass[a4paper,10pt]{article}
\usepackage[utf8]{inputenc}
\usepackage[T1]{fontenc}
\usepackage[french, english]{babel}
\usepackage[hidelinks]{hyperref}
\usepackage{geometry}
\usepackage{enumitem}
\usepackage{xcolor}
\usepackage{titlesec}
\usepackage{parskip} % espace entre paragraphes
\geometry{left=0.5cm,right=0.5cm,top=0.2cm,bottom=0.5cm} % marges réduites

% Couleurs
\definecolor{myblue}{HTML}{003366}

% Format section
\titleformat{\section}{
  \normalfont\large\bfseries\color{myblue}
}{}{0em}{}[\titlerule]

% Commande pour aligner date à droite dans une ligne
\newcommand{\dates}[1]{\hfill {\footnotesize \textit{#1}}}

% Commande pour titre avec lien
\newcommand{\linkedtitle}[2]{\href{#1}{\textbf{\color{myblue}#2}}}

% Réduire espace entre items
\setlist[itemize]{noitemsep, topsep=2pt}

\begin{document}

% ======= NOM =======
\begin{center}
    {\LARGE \textbf{Angel Velasco}}\\[3pt]
    \small
    \href{mailto:angel.velasco0x7@gmail.com}{angel.velasco0x7@gmail.com} \quad
    +33 6 37 73 62 81 \quad
    Yvelines, France \\
    {\Large \href{https://angel0x7.github.io}{Portfolio - https://angel0x7.github.io}} 
\end{center}



% ======= PROFIL =======
\section*{Profil}

\begin{center}
Étudiant ingénieur (ECE) spécialisé en cybersécurité en recherche de stage technique (Blue Team / Sécurité réseau).
\end{center}

% ======= EDUCATION =======
\section*{Éducation}

\noindent
\linkedtitle{https://www.ece.fr/en/program/engineering-cycle-bac4-information-systems-and-cybersecurity-major/}{École Centrale d'Électronique} \hfill Paris, France \\
\emph{Diplôme d'ingénieur – Majeure Cybersécurité – Mineures Défense-Sécurité / International} \hfill Février 2025 -- Présent \\
Spécialisation en cybersécurité défensive développée en partenariat avec Microsoft et Orange Cyberdefense,\\ certifiée SecNumEdu par l’ANSSI. 


\noindent
\linkedtitle{https://www.ece.fr/}{École Centrale d'Électronique} \hfill Paris, France \\
\emph{Cycle préparatoire intégré} \hfill Septembre 2022 -- Juin 2024 \\
Classes préparatoires axées sur les bases théoriques et les compétences informatiques par projets. 


\noindent
\linkedtitle{https://www.dcu.ie/}{Dublin City University} \hfill Dublin, Ireland \\
\emph{Échange International} \hfill Septembre -- Décembre 2024 

% ======= CERTIFICATIONS =======
\section*{Certifications}

\begin{itemize}[leftmargin=*]
    \item Security Blue Team - Blue Team Junior Analyst
    \item Fortinet Certified Fundamentals \& Associate (FCF + FCA) 
    \item DataScientest - Data Analyst
    \item MOOC - Project Management
\end{itemize}

% ======= COMPETENCES =======
\section*{Compétences}

\begin{tabular}{p{5cm} p{15cm}}
\textbf{Cybersécurité \& Réseau} & pfSense (WAN/LAN/DMZ), NAT/port-forward, IDS/IPS, honeypot, Azure Sentinel / KQL. \\
\textbf{Programmation} & Python, Java, C, JavaScript, PHP, SQL. \\
\textbf{Data Science \& ML} & Jupyter, Pandas, scikit-learn (Random Forest, Naive Bayes), TensorFlow (CNN), PyTorch. \\
\textbf{Cloud \& Virtualisation} & Microsoft Azure (VMs, Log Analytics), VirtualBox. \\
\textbf{Outils} & Wireshark, tcpdump, Nessus, Metasploit. \\
\textbf{Web \& BDD} & HTML/CSS, Bootstrap, JavaScript, MySQL, JDBC. \\
\textbf{Gestion de projet} & Scrum, Trello, Jira, planification (Gantt, PERT). \\
\textbf{Langues} & Anglais (B2), Espagnol (natif), Français (natif). \\
\end{tabular}




% ======= COMPETENCES & PROJETS =======
\section*{Projets \& Compétences}

\begin{itemize}[leftmargin=*]
  \item \href{https://github.com/angel0x7/Detection-AI-models-}{\textbf{Detection-AI-models}} \\
  Data Science \& Cybersécurité : Python, Jupyter, NumPy, Scikit-learn, PyTorch, Random Forest, Naive Bayes, détection d’anomalies réseau et spam.

  \item \href{https://github.com/angel0x7/Honeypot-}{\textbf{Honeypot}} \\
  Cybersécurité défensive : déploiement d’un honeypot sur Azure, collecte d’événements dans Microsoft Sentinel, analyse KQL, enrichissement géolocalisé.

  \item \href{https://github.com/angel0x7/Firewall-Lab-WAN-LAN-DMZ-}{\textbf{Firewall-Lab (WAN/LAN/DMZ)}} \\
  Réseaux \& Sécurité : configuration pfSense, segmentation réseau, NAT/Port-forwarding, serveur web IIS, règles de firewall.

  \item \href{https://github.com/angel0x7/AI-NEURAL-SPEECH-}{\textbf{AI-NEURAL-SPEECH}} \\
  Systèmes Embarqués \& IA : C/C++ (Arduino), CNN (TensorFlow), reconnaissance vocale embarquée.


\end{itemize}

\noindent
\textbf{GitHub :} \href{https://github.com/angel0x7}{github.com/angel0x7}

% ======= PROFESSIONAL EXPERIENCE =======
\section*{Expériences Professionnelles}

\begin{itemize}[leftmargin=*]

  \item
  \textbf{\href{https://www.renaultgroup.com/}{Renault Group}} \hfill \emph{Douai, Hauts-de-France, France} \\
  \emph{Technicien de Maintenance} \hfill Janvier 2025 – Février 2025 \\
  Assistance aux opérations de maintenance et  afin d'assurer le bon fonctionnement des équipements de production.

\item
  \textbf{\href{https://www.inmapa.com/en/}{INMAPA}} \hfill \emph{Palencia, Castilla y León, Espagne} \\
  \emph{Opérateur} \hfill Janvier 2024 – Février 2024 \\
  Participation à un projet pour un leader européen de l’aéronautique, supportant les opérations industrielles et le contrôle qualité.

\item
  \textbf{\href{https://www.renaultgroup.com/}{Renault Group}} \hfill \emph{Guyancourt, Île-de-France, France} \\
  \emph{Stagiaire en Ingénieurie Process} \hfill Juillet 2021 – Août 2021 \\
  Contribution aux initiatives d’optimisation des procédés au sein du département ingénierie, améliorant l’efficacité des flux de travail.

    
\end{itemize}

\end{itemize}

\end{document}
